% %!TEX root = ../../document/document.tex

% % For every chapter create a file in the /content/chapters folder with the name
% % chapter-XX.tex where XX is the chapter number  (e.g. 01, 02, 03, ..., 99).


\chapter{Einleitung}
\section*{Motivation}
Die zunehmende Komplexität und Vielfalt betrieblicher Geschäftsprozesse stellt Unternehmen vor die Herausforderung, Abläufe transparent, effizient und nachvollziehbar zu gestalten. Gerade in dynamischen Märkten ist eine strukturierte und konsistente Dokumentation von Prozessen ein zentraler Erfolgsfaktor – sowohl zur Optimierung interner Abläufe als auch zur Erfüllung regulatorischer Anforderungen. Gleichzeitig gewinnt das Thema Workflow-Automatisierung zunehmend an Relevanz, da Unternehmen damit manuelle Tätigkeiten reduzieren, Fehlerquellen minimieren und die Effizienz steigern können. Ein zentrales Ziel dieser Arbeit ist es daher, ein digitales Tool zu entwickeln, das Prozessdokumentation und Workflow-Ausführung in einem flexiblen und gleichzeitig robusten System vereint.

\section*{Problemstellung}
Viele Unternehmen verfügen über keine einheitliche oder systemgestützte Methode zur Prozessdokumentation und -ausführung. Dies führt häufig zu Medienbrüchen, unklaren Zuständigkeiten und einer mangelhaften Nachvollziehbarkeit von Abläufen. Bestehende Softwarelösungen sind entweder zu unflexibel, um unterschiedliche Prozessarten zu unterstützen, oder zu komplex in der Handhabung, um eine breite Akzeptanz im Unternehmensalltag zu erreichen. Zudem fehlt häufig die direkte Verknüpfung zwischen der Prozessmodellierung und deren praktischer Ausführung durch Benutzer. Daraus ergibt sich die Notwendigkeit, ein Tool zu entwickeln, das diese Lücke schließt und sowohl technisch als auch konzeptionell den Anforderungen unterschiedlicher Stakeholder gerecht wird.


\chapter{Aufgabenstellung}
Im Rahmen dieser Arbeit soll ein webbasiertes Tool zur Dokumentation und Ausführung von Geschäftsprozessen konzipiert und prototypisch umgesetzt werden. Ziel ist es, eine Anwendung zu entwickeln, die es ermöglicht, Prozesse strukturiert zu erfassen, visuell darzustellen sowie ihre Ausführung nutzer- und rollenbasiert zu steuern. Dabei sollen grundlegende Funktionen wie das Erstellen, Bearbeiten, Ausführen und Löschen von Prozessen berücksichtigt werden. Die Arbeit umfasst die Analyse der Anforderungen, die Konzeption der Systemarchitektur, die Auswahl geeigneter Technologien sowie die Entwicklung eines lauffähigen Prototyps.

