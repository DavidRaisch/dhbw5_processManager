%!TEX root = ../../document/document.tex

% For every chapter create a file in the /content/chapters folder with the name
% chapter-XX.tex where XX is the chapter number  (e.g. 01, 02, 03, ..., 99).


\chapter{Fazit und Ausblick}
\section*{Fazit}
\todo{Kapitel Fazit schreiben}
% Die entwickelte Process-Manager‑App schließt die definierten Muss‑Anforderungen (Erstellung, Bearbeitung, Löschung, Anzeige und Ausführung von Prozessen sowie rollenbasierte Zugriffssteuerung und Benachrichtigungen) vollständig ab und erfüllt die Soll‑Anforderungen (Filter, Prozessvorschau, Benutzer‑ und Rollenverwaltung) in vollem Umfang. Besonders die Anforderungen an das Rollenmanagement mit Benachrichtigungssystem übersteigt bereits die definierten Anforderungen.

% Insgesamt liefert die Arbeit eine Anwendung für das Management und die Ausführung von Geschäftsprozessen. Das Herzstück bilden der intuitive BPMN‑Editor mit Drag‑\&‑Drop‑Funktionalität und das robuste RBAC‑System. Architekturentscheidungen mit React‑Frontend, Node.js‑Backend und containerisierter Bereitstellung gewährleisten hohe Performance und Skalierbarkeit.

Der entwickelte \textbf{Process‑Manager} schließt die definierten \textbf{Muss‑Anforderungen} (Erstellung, Bearbeitung, Löschung, Anzeige und Ausführung von Prozessen sowie rollenbasierte Zugriffssteuerung und Benachrichtigungen) vollständig ab und erfüllt die \textbf{Soll‑Anforderungen} (Filter in archivierten Instanzen, Prozessvorschau, Benutzer‑ und Rollenverwaltung) in vollem Umfang. Besonders das Rollenmanagement mit seinem fein granulierten Berechtigungskonzept und dem Echtzeit‑Benachrichtigungssystem übertrifft dabei bereits die ursprünglichen Vorgaben.

Das intuitive Zusammenspiel folgender \textbf{Kernkomponenten} macht dies möglich:

\begin{itemize}
  \item \textbf{BPMN‑Editor mit Drag‑\&‑Drop:} Erstellen und Anpassen von Prozessmodellen in wenigen Klicks, inklusive Live‑Vorschau und Validierungsfeedback vor dem Speichern.
  \item \textbf{Archivierte Instanzen mit detaillierten Filtern:} Filtermöglichkeiten über Status, Projekt, Prozess, Ersteller oder Datum erlauben zielgerichtetes Auffinden historischer Prozessdaten.
  \item \textbf{Robustes RBAC‑System:} Rollen lassen sich flexibel definieren, Rechte bis auf Feldebene zuweisen. Automatisierte Benachrichtigungen informieren betroffene Nutzer sofort bei neuen Aufgaben oder Statusänderungen.
  \item \textbf{Performante und skalierbare Architektur:} React‑Frontend und Node.js‑Backend kommunizieren über REST‑APIs; Containerisierung via Docker ermöglicht horizontale Skalierung. MongoDB als NoSQL‑Datenbank garantiert kurze Antwortzeiten auch unter hoher Last.
\end{itemize}

\newpage
\subsection*{Kurzfassung der Analyse der Anforderungen}
\textbf{Muss‑Anforderungen:}
\begin{itemize}
  \item Vollständige Prozessmodellierung (Erstellen, Bearbeiten, Löschen) mit Bestätigungsdialog und Schutz aktiver Instanzen.
  \item Darstellung und Steuerung von Prozessinstanzen in Echtzeit.
  \item Rollenbasierte Zugriffskontrolle (Anlegen, Zuweisen, Entziehen) und automatisierte, konfigurierbare Benachrichtigungen.
\end{itemize}

\textbf{Soll‑Anforderungen:}
\begin{itemize}
  \item Filter in archivierten Instanzen nach Status, Projekt, Prozess, Ersteller und Datum.
  \item Interaktive Prozessvorschau vor Ausführung.
  \item Administrationsoberfläche für Benutzer‑ und Rollenmanagement
\end{itemize}

\textbf{Kann‑Anforderungen:}
\begin{itemize}
  \item Die Kann-Anforderungen sind größtenteils als funktionale Erweiterungen der bestehenden Kernfunktionen konzipiert und bilden eine fundierte Grundlage für die zukünftige Weiterentwicklung des Process Managers. Eine Umsetzung im Rahmen dieser Studienarbeit hätte jedoch den vorgesehenen zeitlichen und inhaltlichen Rahmen überschritten.
\end{itemize}

Damit bietet die Arbeit eine leistungsfähige Plattform für das ganzheitliche Management und die Ausführung von Geschäftsprozessen.


\newpage
\section*{Ausblick}
Im Rahmen der vorliegenden Studienarbeit konnten aufgrund begrenzter zeitlicher und inhaltlicher Ressourcen nicht alle potenziellen Erweiterungen und Funktionalitäten umgesetzt werden. Während der Entwicklung wurde eine lokale Datenbank eingesetzt, da sie für den prototypischen Charakter des Projekts eine einfache und effiziente Lösung darstellt. Diese Datenbank ermöglicht jedoch keine persistente Speicherung über unterschiedliche Systemstarts oder Systemumgebungen hinweg und ist daher für den produktiven Einsatz nicht geeignet.
Für den späteren produktiven Betrieb empfiehlt sich daher die Ablösung der lokalen Datenbank durch eine persistent verfügbare Lösung, beispielsweise eine Cloud-Datenbank oder eine zentral auf einem Firmenserver gehostete Datenbank. Welche Variante gewählt wird, hängt maßgeblich von zukünftigen Anforderungen und Rahmenbedingungen im produktiven Einsatz ab – insbesondere im Hinblick auf Skalierbarkeit, Zugriffssicherheit und Integration in bestehende Systemlandschaften. Da diese Anforderungen zum Zeitpunkt der Studienarbeit nicht definiert werden können, wird auf eine frühzeitige Integration einer solchen Lösung bewusst verzichtet. Die lokale Datenbank verbleibt somit als temporäre Lösung im Rahmen der Entwicklungsumgebung. Es ist ausdrücklich darauf hinzuweisen, dass diese nicht für den produktiven Betrieb vorgesehen ist.

Darüber hinaus sind weitere Funktionalitäten identifiziert worden, die potenziell einen Mehrwert für die Benutzerfreundlichkeit und Erweiterbarkeit des Systems bieten würden. So könnte beispielsweise eine Funktion implementiert werden, die es dem Benutzer erlaubt, neue Prozesse über die grafische Benutzeroberfläche zu laden. Dabei würde die zugehörige XML-Datei direkt in die Datenbank übernommen und der neue Prozess würde anschließend zur Bearbeitung zur Verfügung stehen.

Ein weiteres Beispiel ist die Möglichkeit, den Drag-and-Drop-Arbeitsbereich auf Knopfdruck zu vergrößern, sodass der Nutzer den gesamten Bildschirm zur Bearbeitung von Prozessen nutzen kann. Solche Erweiterungen würden die Interaktion mit dem System verbessern und die Effizienz bei der Prozessmodellierung erhöhen.

Diese Beispiele verdeutlichen, dass Softwareprojekte grundsätzlich einem iterativen Entwicklungsprozess unterliegen. Sie werden durch Rückmeldungen von Anwendern und sich ändernde Anforderungen kontinuierlich weiterentwickelt. Auch das hier vorgestellte System bietet Potenzial für zukünftige Funktionserweiterungen, die im Rahmen weiterer Arbeiten oder eines produktiven Einsatzes realisiert werden könnten.

\todo{Komplette Arbeit nochmal lesen}



